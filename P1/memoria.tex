\documentclass[11pt,a4paper]{article}
\usepackage[spanish,es-nodecimaldot]{babel}	% Utilizar español
\usepackage[utf8]{inputenc}					% Caracteres UTF-8
\usepackage{graphicx}						% Imagenes
\usepackage[hidelinks]{hyperref}			% Poner enlaces sin marcarlos en rojo
\usepackage{fancyhdr}						% Modificar encabezados y pies de pagina
\usepackage{float}							% Insertar figuras
\usepackage[textwidth=390pt]{geometry}		% Anchura de la pagina
\usepackage[nottoc]{tocbibind}				% Referencias (no incluir num pagina indice en Indice)
\usepackage{enumitem}						% Permitir enumerate con distintos simbolos
\usepackage[T1]{fontenc}					% Usar textsc en sections
\usepackage{amsmath}						% Símbolos matemáticos
\usepackage{seqsplit}						% Escribir numeros muy largos

\usepackage{listings}
\usepackage{xcolor}

\definecolor{codegreen}{rgb}{0,0.6,0}
\definecolor{codegray}{rgb}{0.5,0.5,0.5}
\definecolor{codepurple}{rgb}{0.58,0,0.82}
\definecolor{backcolour}{rgb}{0.95,0.95,0.92}

\lstdefinestyle{mystyle}{
    backgroundcolor=\color{backcolour},   
    commentstyle=\color{codegreen},
    keywordstyle=\color{magenta},
    numberstyle=\tiny\color{codegray},
    stringstyle=\color{codepurple},
    basicstyle=\ttfamily\footnotesize,
    breakatwhitespace=false,         
    breaklines=true,                 
    captionpos=b,                    
    keepspaces=true,                 
    numbers=left,                    
    numbersep=5pt,                  
    showspaces=false,                
    showstringspaces=false,
    showtabs=false,                  
    tabsize=4,
    language=Python
}

\lstset{style=mystyle}

% Comando para poner el nombre de la asignatura
\newcommand{\asignatura}{Criptografía y Computación}
\newcommand{\autor}{Vladislav Nikolov Vasilev}
\newcommand{\titulo}{Práctica 1}
\newcommand{\subtitulo}{Primalidad}
\newcommand{\rama}{Computación y Sistemas Inteligentes}

% Configuracion de encabezados y pies de pagina
\pagestyle{fancy}
\lhead{\autor{}}
\rhead{\asignatura{}}
\lfoot{Grado en Ingeniería Informática}
\cfoot{}
\rfoot{\thepage}
\renewcommand{\headrulewidth}{0.4pt}		% Linea cabeza de pagina
\renewcommand{\footrulewidth}{0.4pt}		% Linea pie de pagina


% Comando para seccion sin numero
\newcommand{\nonumsection}[1]{\section*{#1}\addcontentsline{toc}{section}{#1}}

% Comando para mostrar comandos de shell
\newcommand{\shellcmd}[1]{\indent\indent\texttt{\footnotesize\$ #1}\\}

\begin{document}
\pagenumbering{gobble}

% Pagina de titulo
\begin{titlepage}

\begin{minipage}{\textwidth}

\centering

%\includegraphics[scale=0.5]{img/ugr.png}\\
\includegraphics[scale=0.3]{img/logo_ugr.jpg}\\[1cm]

\textsc{\Large \asignatura{}\\[0.2cm]}
\textsc{GRADO EN INGENIERÍA INFORMÁTICA}\\[1cm]

\noindent\rule[-1ex]{\textwidth}{1pt}\\[1.5ex]
\textsc{{\Huge \titulo\\[0.5ex]}}
\textsc{{\Large \subtitulo\\}}
\noindent\rule[-1ex]{\textwidth}{2pt}\\[3.5ex]

\end{minipage}

%\vspace{0.5cm}
\vspace{0.7cm}

\begin{minipage}{\textwidth}

\centering

\textbf{Autor}\\ {\autor{}}\\[2.5ex]
\textbf{Rama}\\ {\rama}\\[2.5ex]
\vspace{0.3cm}

\includegraphics[scale=0.3]{img/etsiit.jpeg}

\vspace{0.7cm}
\textsc{Escuela Técnica Superior de Ingenierías Informática y de Telecomunicación}\\
\vspace{1cm}
\textsc{Curso 2019-2020}
\end{minipage}
\end{titlepage}

\pagenumbering{arabic}
\tableofcontents
\thispagestyle{empty}				% No usar estilo en la pagina de indice

\newpage

\setlength{\parskip}{1em}

\nonumsection{Instrucciones de ejecución}

Se ha adjuntado un \textit{script} de Python que ejecuta cada uno de los ejercicios
y muestra los resultados. Para ejecutarlo se necesita tener \texttt{python3} instalado en el
equipo, y basta con situarse en el lugar donde se tenga el \textit{script} y ejecutar:

\shellcmd{python3 primalidad.py}

La ejecución tarda poco más de 2 minutos debido a que en uno de los ejercicios
se hace un cómputo bastante costoso (se calcula un primo fuerte de $500$ bits).

\nonumsection{Ejercicio 1}

En este ejercicio se pide implementar una función que realice el test de Miller-Rabin
dados un número impar $n$ y un testigo $a$ tal que $2 \leq a \leq n-2$. La función
debe devolver verdadero en caso de que $n$ sea probable primo y falso en caso contrario.

Por una parte, para realizar el test de Miller-Rabin necesitamos una función que
calcule la descomposición de $n-1$ como $2^u * s$, donde $s$ es un número impar. Esta
función se ha implementado de la siguiente forma:

\begin{lstlisting}
def descomposicion(n):
    # Inicializar u y s
    u = 0
    s = n
    
    while s % 2 == 0:
        u += 1
        s = s // 2
    
    return u, s
\end{lstlisting}

La función que realiza el test de Miller-Rabin para un $n$ y un $a$ dados es la siguiente:

\begin{lstlisting}
def miller_rabin(n, a):
    # 1. Descomponer n-1 como 2^u * s con s impar
    u, s = descomposicion(n-1)
    
    # 2. Calcular a = a^s mod n
    a = potencia_modular(a, s, n)
    
    # Si a == 1 o a == n-1, el numero es posible primo
    if a == 1 or a == n-1:
        return True
    
    for i in range(1, u):
        a = potencia_modular(a, 2, n)
        
        # Si a == 1 sin haber pasado por n-1, el numero no es primo
        # ya que tiene mas de una solucion a x^2 - 1 = 0
        if a == 1:
            return False
        
        """
        Si a == n-1, el siguiente valor sera 1, por lo tanto,
        cumpliria el test de Fermat y tendria solo dos soluciones a
        la ecuacion x^2 - 1 = 0. Puede ser primo
        """
        if a == n-1:
            return True
    
    return False
\end{lstlisting}

Se ha probado la función anterior con $n = 1729$ y con dos testigos: $a_1 = 2$ y $a_2 = 10$.
En el primero caso, la función ha determinado que $n$ no es primo, mientras que en el segundo
caso ha determinado que sí que lo es. Este comportamiento es el esperado, ya que sabemos que
$1729 = 7 \cdot 247$ y que por tanto no es primo, y que $a = 10$ es un falso testigo.

\nonumsection{Ejercicio 2}

En este ejercicio se ha pedido que se implemente una función que realice el test de Miller-Rabin
escogiendo $m$ testigos aleatorios. La función es la siguiente:

\begin{lstlisting}
def test_primalidad(n, m):
    for i in range(m):
        # Escoger testigo tal que 2 <= a <= n-2
        a = random.randint(2, n-2)
        
        es_prob_primo = miller_rabin(n, a)
        
        if not es_prob_primo:
            return False
    
    return True
\end{lstlisting}

En el momento en el que el test de Miller-Rabin devuelva falso, se ha conseguido determinar
que el número no es probable primo.

Se ha probado esta función con tres números y con $m = 20$, ya que con dicho valor
nos podemos asegurar que la probabilidad de que falle el test de primalidad sea menor
a $\frac{1}{4^{20}}$. Los números probados han sido $n_1 = 341$, $n_2 = 1729$ y $n_3 = \seqsplit{%
203956878356401977405765866929034577280193993314348263094772646453283062722701277632936616063144088173312372882677123879538709400158306567338328279154499698366071906766440037074217117805690872792848149112022286332144876183376326512083574821647933992961249917319836219304274280243803104015000563790123}$. De estos tres números, solo $n_3$ es primo.

Los resultados que ha ofrecido la función una vez que ha sido ejecutada han sido correctos,
ya que ha dicho que los dos primeros números no son primos y que el tercero es probable
primo.

\nonumsection{Ejercicio 3}

En este ejercicio se pide implementar una función que dado un número $n$ se calcule un número
$n'$ tal que $n \leq n'$ y $n'$ sea probable primo.

Para determinar dicho número nos podemos ayudar de la función anterior. Podemos ir recorriendo
los números a partir de $n$ y hacerles un test de primalidad, y en el momento en el que nos
encontremos con un probable primo, devolverlo. La funcionalidad anterior se ha implementado
de la siguiente forma:

\begin{lstlisting}
def siguiente_primo(n, m):
    es_posible_primo = False

    while not es_posible_primo:
        es_posible_primo = test_primalidad(n, m)

        if es_posible_primo:
            posible_primo = n
        
        n += 1
    
    return posible_primo
\end{lstlisting}

Para ver si la función es correcta se ha probado con $n_1 = 14$ y $n_2 = 1729$. En el primer
caso se ha obtenido $n'_1 = 17$, mientras que en el segundo se ha obtenido $n'_2 = 1733$. Ambos
números son primos (aparecen en cualquier lista de números primos que se pueda encontrar
por internet), y por tanto, el funcionamiento parece correcto.

\nonumsection{Ejercicio 4}

En este ejercicio se pide implementar una función que dado un número $n$ encuentre el primer
probable primo fuerte $n'$ tal que $n \leq n'$. El número $n'$ es primo fuerte si tanto él como
$\frac{n'-1}{2}$ son primos.

Para hacerlo, se han implementado las siguientes funciones:

\begin{lstlisting}
def test_primo_fuerte(n, m):
    return test_primalidad((n - 1) // 2, m)


def siguiente_primo_fuerte(n, m):
    es_primo_fuerte = False

    while not es_primo_fuerte:
        n = siguiente_primo(n, m)
        es_primo_fuerte = test_primo_fuerte(n, m)

        if es_primo_fuerte:
            primo_fuerte = n
        
        n += 1
    
    return primo_fuerte
\end{lstlisting}

La función va buscando primos probables y cada vez que se topa con uno intenta determinar
si es un pirmo fuerte. En caso de que lo sea, lo devuelve, y en caso contrario, continua
con la búsqueda.

Se ha probado la función con $n_1 = 12$ y con $n_2 = 1729$. En el primer caso, el primer
primo fuerte encontrado ha sido $23$, mientras que en el segundo ha sido $1823$. Con la ayuda
de una tabla de primos que se puede encontrar en internet se han comprobado los resultados y
se ha visto que son correctos. En el primer caso esto es así porque tanto $23$ como $11$ son
primos, mientras que en el segundo porque tanto $1823$ como $911$ son primos. Por tanto, la
función parece tener el comportamiento esperado.

\nonumsection{Ejercicio 5}

En este apartado se ha pedido implementar una función que calcule el primer probable primo
fuerte de $n$ bits. Este primo, $p$, al tener $n$ bits, tendrá que cumplir que su valor esté
en el rango $2^{n-1} \leq p \leq 2^n - 1$.

La función que se ha implementado es la siguiente:

\begin{lstlisting}
def primo_fuerte_n_bits(n, m):
    return siguiente_primo_fuerte(2 ** (n-1), m)
\end{lstlisting}

Se ha probado la función anterior con una serie de valores de $n$ y se han obtenido los
siguientes resultados:

\begin{itemize}
	\item Con $n = 10$ bits se ha obtenido que el primer primo fuerte es $563$.
	\item Con $n = 25$ bits se ha obtenido que el primer primo fuerte es $16777907$.
	\item Con $n = 50$ bits se ha obtenido que el primer primo fuerte es $562949953422839$.
	\item Con $n = 100$ bits se ha obtenido que el primer primo fuerte es
	$\seqsplit{633825300114114700748351612867}$.
	\item Con $n = 500$ bits se ha obtenido que el primer primo fuerte es
	$\seqsplit{1636695303948070935006594848413799576108321023021532394741645684048066898202337277441635046162952078575443342063780035504608628272942696526664264070799}$.
\end{itemize}

Cada resultado se ha pasado por un factorizador en línea y se ha comprobado que todos cumplen
las condiciones para ser primos fuertes. Por tanto, la función tiene el comportamiento esperado. 

\nonumsection{Ejercicio 6}

En este ejercicoi se deben escoger tres números compuestos $n_1$, $n_2$ y $n_3$, los cuáles
deben cumplir una serie de restricciones. En el caso de $n_1$ se tienen que obtener todos los
falsos testigos, mientras que para $n_2$ y $n_3$ se deben probar $200$ testigos aleatorios
y determinar todos los falsos testigos.

Para calcular todos los falsos testigos de un número se ha utlizado la siguiente
función:

\begin{lstlisting}
def calcular_todos_falsos_testigos(n):
    falsos_testigos = []

    for a in range(2, n - 1):
        if miller_rabin(n, a):
            falsos_testigos.append(a)
    
    return falsos_testigos
\end{lstlisting}

Para probar una serie de $m$ testigos y determinar los falsos testigos se ha utilizado la
siguiente función:

\begin{lstlisting}
def calcular_m_falsos_testigos(n, m):
    falsos_testigos = []

    for _ in range(m):
        a = random.randint(2, n - 2)

        if miller_rabin(n, a):
            falsos_testigos.append(a)

    return falsos_testigos
\end{lstlisting}

Finalmente, para calcular la proporción de falsos testigos se ha utilizado la siguiente
función:

\begin{lstlisting}
def calcular_proporcion_falsos_testigos(falsos_testigos, m):
    return len(falsos_testigos) / m
\end{lstlisting}

Una vez que se han visto las funciones a utilizar, vamos a pasar a ver qué números compuestos se
han escogido y los resultados que se han obtenido.

Se ha escogido que $n_1$ sea $n_1 = 11^2 = 121$. Los falsos testigos que se han encontrado han
sido 3, 9, 27, 40, 81, 94, 112 y 118, lo cuál representa una proporción de aproximadamente $0.06$
de todos los posibles testigos.

Para $n_2$ se ha escogido que su valor sea
$n_2 = 11 \cdot 13 \cdot 17 \cdot 19 \cdot 23 = 1062347$. En este caso no se ha obtenido ningún
falso testigo.

Para $n_3$ se han escogido un primo fuerte mayor que $10000000$ y otro fuerte de $25$ bits
y se han multiplicado. El valor de $n_3$ obtenido ha sido
$n_3 = 10000223 \cdot 16777907 = 167782811473261$. Nuevamente tampoco se han encontrado falsos
testigos.

\nonumsection{Ejercicio 7}

En este ejercicio se pedía que se probasen 200 testigos aleatorios y que se determinasen
cuáles de ellos habían sido falsos testigos para el número $n = 3215031751$.

Los falsos testigos encontrados han sido los siguientes:

\noindent 2989642428, 2543428172, 614042734, 1424011350, 300843473, 3183414300, 2410287500, 2962532651, 419843896, 1715915166, 2241650903, 1889534559, 3105476637, 516122673, 2109566925, 901700272, 1197571907, 2231960901, 3128587513, 3148414053, 3034801065, 1955701261, 327244541, 2216210500, 1845860340, 1084176316, 2012890817, 313758577, 515904415, 959186557, 2272308219, 1423678299, 3202777707, 750340098, 1310849420, 508705234, 2108513998, 1171046710, 2130543965, 2324710841, 2050129978, 666132191

De los 200 testigos probados, un total de 42 de ellos han resultado ser falsos testigos,
lo cuál representa una proporción de $0.21$.

\nonumsection{Ejercicio 8}

En este ejercicio se ha pedido que se escogiesen 100 testigos aleatorios para el número
$n = 2199733160881$ y se determinase cuáles y cuántos de ellos resultaban ser falsos testigos
según el test de Fermat y el de Miller-Rabin.

La función que se ha implementado para hacer esto se puede ver a continuación:

\begin{lstlisting}
def falsos_testigos_fermat_miller_rabin(n, m):
    falsos_testigos_fermat = []
    falsos_testigos_miller_rabin = []

    for _ in range(m):
        a = random.randint(2, n - 2)

        if test_fermat(n, a):
            falsos_testigos_fermat.append(a)
        
        if miller_rabin(n, a):
            falsos_testigos_miller_rabin.append(a)

    return falsos_testigos_fermat, falsos_testigos_miller_rabin
\end{lstlisting}

Los falsos testigos que se han encontrado mediante el test de Fermat han sido los siguientes:

\noindent 541294428553, 427414670640, 856840996708, 869459582987, 1386150924507, 2021284951395, 1049180344086, 1436874409098, 2080622412228, 1895446961219, 607589646504, 95738089244, 500152881760, 273343767955, 816294840698, 2137304862156, 1686099093795, 105112798591, 76108696568, 2084939446410, 288639221475, 2175697704600, 1521166120712, 1208550220361, 734596648309, 1775915927632, 1444025072928, 1539558933563, 317159111698, 153191386359, 107087172632, 626790072879, 1721765538668, 1339295283967, 309820597244, 1499387178030, 645650661512, 251351171707, 125102898679, 670739616035, 1664579380, 1166105943589, 1623944880881, 1439812021977, 1291686686494, 1254049709277, 973102251046, 697908434478, 1154053174720, 1931057820368, 645605063013, 348148721167, 1467216861092, 2100887387037, 1196824929484, 506736084768, 382854241647, 1083675018117, 157094190622, 770574992, 611934573350, 1723351143188, 1748441690699, 362412962126, 443665136244, 1518280794907, 1494524424336, 278756486397, 2059386093625, 545388027793, 851959036442, 1992276265019, 2195305341090, 399547067632, 89834049489, 696316386402, 589834218012, 557877780581, 62877284258, 2071701656149, 1813750085866, 199765571834, 696380393864, 873234754041, 10738916652, 925428820553, 1057060786293, 2009802064885, 257038689587, 277003366590, 531869096808, 1194918230399, 581254591237, 337015700018, 1190017832155, 1329585096763, 1955831589860, 137053152985, 1154693933317, 363634170708

Los falsos testigos encontrados mediante el test de Miller-Rabin han sido los siguientes:

\noindent 1686099093795, 105112798591, 697908434478, 1518280794907, 589834218012, 199765571834, 277003366590

En este caso particular, todos los testigos probados han resultado ser falsos testigos utilizando
el test de Fermat, mientras que solo 7 de ellos han resultado serlo con el de Miller-Rabin.

\end{document}

