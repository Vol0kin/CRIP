\documentclass[11pt,a4paper]{article}
\usepackage[spanish,es-nodecimaldot]{babel}	% Utilizar español
\usepackage[utf8]{inputenc}					% Caracteres UTF-8
\usepackage{graphicx}						% Imagenes
\usepackage[hidelinks]{hyperref}			% Poner enlaces sin marcarlos en rojo
\usepackage{fancyhdr}						% Modificar encabezados y pies de pagina
\usepackage{float}							% Insertar figuras
\usepackage[textwidth=390pt]{geometry}		% Anchura de la pagina
\usepackage[nottoc]{tocbibind}				% Referencias (no incluir num pagina indice en Indice)
\usepackage{enumitem}						% Permitir enumerate con distintos simbolos
\usepackage[T1]{fontenc}					% Usar textsc en sections
\usepackage{amsmath}						% Símbolos matemáticos

\usepackage{listings}
\usepackage{xcolor}

\definecolor{codegreen}{rgb}{0,0.6,0}
\definecolor{codegray}{rgb}{0.5,0.5,0.5}
\definecolor{codepurple}{rgb}{0.58,0,0.82}
\definecolor{backcolour}{rgb}{0.95,0.95,0.92}

\lstdefinestyle{mystyle}{
    backgroundcolor=\color{backcolour},   
    commentstyle=\color{codegreen},
    keywordstyle=\color{magenta},
    numberstyle=\tiny\color{codegray},
    stringstyle=\color{codepurple},
    basicstyle=\ttfamily\footnotesize,
    breakatwhitespace=false,         
    breaklines=true,                 
    captionpos=b,                    
    keepspaces=true,                 
    numbers=left,                    
    numbersep=5pt,                  
    showspaces=false,                
    showstringspaces=false,
    showtabs=false,                  
    tabsize=4,
    language=Python
}

\lstset{style=mystyle}

% Comando para poner el nombre de la asignatura
\newcommand{\asignatura}{Criptografía y Computación}
\newcommand{\autor}{Vladislav Nikolov Vasilev}
\newcommand{\titulo}{Práctica 1}
\newcommand{\subtitulo}{Primalidad}
\newcommand{\rama}{Computación y Sistemas Inteligentes}

% Configuracion de encabezados y pies de pagina
\pagestyle{fancy}
\lhead{\autor{}}
\rhead{\asignatura{}}
\lfoot{Grado en Ingeniería Informática}
\cfoot{}
\rfoot{\thepage}
\renewcommand{\headrulewidth}{0.4pt}		% Linea cabeza de pagina
\renewcommand{\footrulewidth}{0.4pt}		% Linea pie de pagina


% Comando para seccion sin numero
\newcommand{\nonumsection}[1]{\section*{#1}\addcontentsline{toc}{section}{#1}}

\begin{document}
\pagenumbering{gobble}

% Pagina de titulo
\begin{titlepage}

\begin{minipage}{\textwidth}

\centering

%\includegraphics[scale=0.5]{img/ugr.png}\\
\includegraphics[scale=0.3]{img/logo_ugr.jpg}\\[1cm]

\textsc{\Large \asignatura{}\\[0.2cm]}
\textsc{GRADO EN INGENIERÍA INFORMÁTICA}\\[1cm]

\noindent\rule[-1ex]{\textwidth}{1pt}\\[1.5ex]
\textsc{{\Huge \titulo\\[0.5ex]}}
\textsc{{\Large \subtitulo\\}}
\noindent\rule[-1ex]{\textwidth}{2pt}\\[3.5ex]

\end{minipage}

%\vspace{0.5cm}
\vspace{0.7cm}

\begin{minipage}{\textwidth}

\centering

\textbf{Autor}\\ {\autor{}}\\[2.5ex]
\textbf{Rama}\\ {\rama}\\[2.5ex]
\vspace{0.3cm}

\includegraphics[scale=0.3]{img/etsiit.jpeg}

\vspace{0.7cm}
\textsc{Escuela Técnica Superior de Ingenierías Informática y de Telecomunicación}\\
\vspace{1cm}
\textsc{Curso 2019-2020}
\end{minipage}
\end{titlepage}

\pagenumbering{arabic}
\tableofcontents
\thispagestyle{empty}				% No usar estilo en la pagina de indice

\newpage

\setlength{\parskip}{1em}

\nonumsection{Instrucciones de ejecución}

\nonumsection{Ejercicio 1}

En este ejercicio se pide implementar una función que realice el test de Miller-Rabin
dados un número impar $n$ y un testigo $a$ tal que $2 \leq a \leq n-2$. La función
debe devolver verdadero en caso de que $n$ sea probable primo y falso en caso contrario.

Por una parte, para realizar el test de Miller-Rabin necesitamos una función que
calcule la descomposición de $n-1$ como $2^u * s$, donde $s$ es un número impar. Esta
función se ha implementado de la siguiente forma:

\begin{lstlisting}
def descomposicion(n):
    # Inicializar u y s
    u = 0
    s = n
    
    while s % 2 == 0:
        u += 1
        s = s // 2
    
    return u, s
\end{lstlisting}

La función que realiza el test de Miller-Rabin para un $n$ y un $a$ dados es la siguiente:

\begin{lstlisting}
def miller_rabin(n, a):
    # 1. Descomponer n-1 como 2^u * s con s impar
    u, s = descomposicion(n-1)
    
    # 2. Calcular a = a^s mod n
    a = potencia_modular(a, s, n)
    
    # Si a == 1 o a == n-1, el numero es posible primo
    if a == 1 or a == n-1:
        return True
    
    for i in range(1, u):
        a = potencia_modular(a, 2, n)
        
        # Si a == 1 sin haber pasado por n-1, el numero no es primo
        # ya que tiene mas de una solucion a x^2 - 1 = 0
        if a == 1:
            return False
        
        """
        Si a == n-1, el siguiente valor sera 1, por lo tanto,
        cumpliria el test de Fermat y tendria solo dos soluciones a
        la ecuacion x^2 - 1 = 0. Puede ser primo
        """
        if a == n-1:
            return True
    
    return False
\end{lstlisting}

Se ha probado la función anterior con $n = 1729$ y con dos testigos: $a_1 = 2$ y $a_2 = 10$.
En el primero caso, la función ha determinado que $n$ no es primo, mientras que en el segundo
caso ha determinado que sí que lo es. Este comportamiento es el esperado, ya que sabemos que
$1729 = 7 \cdot 247$ y que por tanto no es primo, y que $a = 10$ es un falso testigo.

\nonumsection{Ejercicio 2}

En este ejercicio se ha pedido que se implemente una función que realice el test de Miller-Rabin
escogiendo $m$ testigos aleatorios. La función es la siguiente:

\begin{lstlisting}
def test_primalidad(n, m):
    for i in range(m):
        # Escoger testigo tal que 2 <= a <= n-2
        a = random.randint(2, n-2)
        
        es_prob_primo = miller_rabin(n, a)
        
        if not es_prob_primo:
            return False
    
    return True
\end{lstlisting}

En el momento en el que el test de Miller-Rabin devuelva falso, se ha conseguido determinar
que el número no es probable primo.

\nonumsection{Ejercicio 3}

\nonumsection{Ejercicio 4}

\nonumsection{Ejercicio 5}

\nonumsection{Ejercicio 6}

\nonumsection{Ejercicio 7}

\nonumsection{Ejercicio 8}

\end{document}

